 \documentclass[12pt]{article}
%% note that some of the references from the answers were changed by hand
%% search for Section and Figure
  \usepackage{wrapfig}
 \input{mystyle.sty}
    \usepackage{proof}
    \usepackage{url}
    \usepackage{fitch}
    \usepackage[all]{xy}  
      \usepackage{enumitem}
%\thispagestyle{empty}
\usepackage{tikz}
\usetikzlibrary{arrows,shapes,snakes,automata,backgrounds,petri,trees}
\usepackage{pifont}
\usepackage{pxfonts} 
\oddsidemargin 0pt
\evensidemargin 0 pt
\topmargin -.3in
\headsep 20pt
\footskip 20pt
\textheight 8.5in
\textwidth 6.25in
%\newcommand{\cal}[1]{\mathcal{#1}}
% Alter some LaTeX defaults for better treatment of figures:
    % See p.105 of "TeX Unbound" for suggested values.
    % See pp. 199-200 of Lamport's "LaTeX" book for details.
    %   General parameters, for ALL pages:
%    \renewcommand{\truefraction}{0.9}	% max fraction of floats at top
%    \renewcommand{\falsetomfraction}{0.8}	% max fraction of floats at bottom
    %   Para meters for TEXT pages (not float pages):
    \setcounter{topnumber}{2}
    \setcounter{bottomnumber}{2}
    \setcounter{totalnumber}{4}     % 2 may work better
    \setcounter{dbltopnumber}{2}    % for 2-column pages
    \renewcommand{\dbltopfraction}{0.9}	% fit big float above 2-col. text
    \renewcommand{\textfraction}{0.07}	% allow minimal text w. figs
    %   Parameters for FLOAT pages (not text pages):
    \renewcommand{\floatpagefraction}{0.7}	% require fuller float pages
	% N.B.: floatpagefraction MUST be less than topfraction !!
    \renewcommand{\dblfloatpagefraction}{0.7}	% require fuller float pages


\renewcommand\thesubsection{\arabic{subsection}}

 \begin{document}


\begin{center}
{
\Large  Modal Logic, Winter 2019   \\
   Homework 5
\\
due Tuesday, February 12 \\
} 
\end{center}

{\bf Note} In this homework, and in all work in this class: unless otherwise stated,
you may use the soundness or completeness of a proof system.
But whenever you do, you need to say which one you are using.


\section{de Morgan's Laws}
 

\begin{enumerate}
\item Give a derivation that shows $\proves\nott(\phi\andd\psi)\iif (\nott \phi)\orr (\nott\psi)$.
\item Give a derivation that shows $\proves(\nott \phi)\orr (\nott\psi) \iif \nott(\phi\andd\psi)$.
\end{enumerate}
In one of the two parts, you will need to use the Law of the Excluded Middle.

\section{Satisfiability in propositional logic}
Recall that  a propositional logic sentence
$\phi$ is \emph{satisfiable} if there is some valuation $v$ such that $\semantics{$\phi$}_v =\true$.

For each of the following sentences, tell whether true
or false.  For the true ones, give a short proof.
For the false ones, give a counterexample.
\begin{enumerate}
\index{satisfiable!in propositional logic}
\item Every sentence $\phi$ or its negation $\nott\phi$ is satisfiable.
\item Both $\phi$ and $\nott\phi$ are satisfiable. \label{partb}
\item If $\phi\andd\psi$ is satisfiable, then both $\phi$ and $\psi$ are satisfiable. 
\item If  both $\phi$ and $\psi$ are satisfiable,
then $\phi\andd\psi$ is satisfiable.
\item If $\phi\orr\psi$ is satisfiable, then either $\phi$ or $\psi$
(or both) are satisfiable. 
\item If  either $\phi$ or $\psi$ is satisfiable,
then $\phi\orr\psi$ is satisfiable.
\item  If $\phi$ and $\phi\iif\psi$ are satisfiable, then $\psi$ is satisfiable.
\item Every sentence $\phi$ or its negation $\nott\phi$ is a tautology.
\item  If $\phi\andd\psi$ is a tautology, then both $\phi$ and $\psi$ are tautologies. 
\item If $\phi\orr\psi$ is a tautology, then either $\phi$ or $\psi$
(or both) are tautologies. 
\item If $\phi$ and $\phi\iif\psi$ are tautologies, then $\psi$ is a tautology.
\end{enumerate}
[As a hint, I'll solve the first two parts.
Part (1) is true for the following reason: 
Fix the sentence $\phi$.   Let $v$ be any valuation.
Either $\semantics{$\phi$}  = \true$ or $\semantics{$\phi$} = \false$.
In the first case, $v$ shows that $\phi$ is satisfiable.
In the second case, 
$$\semantics{$\nott\phi$}  = \nott \semantics{$\phi$}  = \nott \false  = \true,$$
and therefore $v$ shows that $\nott\phi$ is satisfiable.
On the other hand, part (\ref{partb}) is false.
We may take $\phi$ to be $p\andd\nott p$.  This sentence is not satisfiable.]

 


\section{Avoiding confusion}

It is easy to confuse 
the following two assertions:
\begin{enumerate}
\item[(i)]   $\not\proves \phi\iif \psi$ 
\item[(ii)]  $\proves \phi\iif \nott\psi$.
\end{enumerate}
(i) says that there is \emph{no} derivation in our system that has no premises and ends with $\phi\iif \psi$.
(ii) says that there  \emph{is} a  derivation in our system that has no premises and ends with $ \phi\iif \nott\psi$.

\begin{enumerate}
\item Give an example of two sentences $\phi$ and $\psi$ in propositional logic 
with the property that $\not\proves \phi\iif \psi$ but    $ \not\proves \phi\iif \nott\psi$.
This shows that (i) does not in general imply (ii).
\item Give an example of two sentences $\phi$ and $\psi$ in propositional logic 
with the property that $\proves \phi\iif \psi$ and also   $ \proves \phi\iif \nott\psi$.
This shows that (ii) does not in general imply (i).
\end{enumerate}

 
\section{A step in the lemma on state descriptions}
Recall from class the main lemma on state descriptions.
It says:   For all sentences $\phi$, 
and all state descriptions $\alpha$, 
if $\occ(\phi)\subseteq \occ(\alpha)$, then 
  either $\proves \alpha\iif \phi$, or else $\proves \alpha\iif \nott\phi$.

The proof was by induction on $\phi$.  In the lecture slides, you can find the induction steps for $\andd$ and for $\nott$.
Your task: prove the induction step for $\iif$.
Be sure to use the relation between the sets $\occ(\phi\iif\psi)$, $\occ(\phi)$, and $occ(\psi)$.


 \section{Consistent and satisfiable}
 
 A sentence $\phi$ in propositional logic  is \emph{consistent} if $\not\proves\nott \phi$.
That is, $\nott\phi$ is \emph{not} provable.  

\begin{enumerate}
\item 
Prove that if $\phi$ is consistent, then $\phi$ is satisfiable.
\item
Prove that if $\phi$ is satisfiable, then $\phi$ is consistent.
\end{enumerate}
You will need to use either the soundness or completeness of our logic, or both.
    Please be sure to write down exactly where
you used these results.



\end{document}
