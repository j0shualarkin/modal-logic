\documentclass{article}
\usepackage{inputenc,amsmath,amsthm,amssymb,amsfonts,tikz}

\title{M482 Homework 3}
\author{yr name here}
\date{Tuesday, 29 January 2019}

\begin{document}

\maketitle

\section{Satisfiability in modal logic}
A modal sentence $\phi$ is $\textit{satisfiable}$ if there is some model $M$ and some $x \in M$ such that $x \models \phi$. Show that each of the following sentences is satisfiable.
\begin{enumerate}
    \item $(\Diamond\Box p) \land (\Box\Diamond p)$
    \item $(\Diamond\Box p) \land (\neg\Box\Diamond p)$
    \item $(\neg\Diamond\Box p) \land (\Box\Diamond p)$
    \item $(\neg\Diamond\Box p) \land (\neg\Box\Diamond p)$
\end{enumerate}


\vspace{1.5in}

\section{Satisfiability on a symmetric model}
A modal sentence $\phi$ is $\textit{satisfiable on a symmetric model}$ if there is some $\textit{symmetric model} M$ and some $x \in M$ such that $x \models \phi$. Which of the following sentences is satisfiable on a symmetric model?

\begin{enumerate}
    \item $(\Diamond\Box p) \land (\Box\Diamond p)$
    \item $(\Diamond\Box p) \land (\neg\Box\Diamond p)$
    \item $(\neg\Diamond\Box p) \land (\Box\Diamond p)$
    \item $(\neg\Diamond\Box p) \land (\neg\Box\Diamond p)$
\end{enumerate}

\vspace{1.5in}

\section{Partitions on a fixed set $X$, and an order on them}

\begin{enumerate}
    \item Find five partitions of $\{1,2,3,4\}$. There are fifteen in total, but you only need to find five of them. Call your partitions $\Pi_1 , \dots , \Pi_5$.
    
    \begin{enumerate}
        \item $\Pi_1$
        \item $\Pi_2$
        \item $\Pi_3$
        \item $\Pi_4$
        \item $\Pi_5$
    \end{enumerate}
    
    \item Let $X$ be a set. Let $\Pi$ and $\Pi '$ be partitions on $X$. We say that $\Pi ' \textit{ refines } \Pi$ if every cell of $\Pi '$ is a subset of a cell of $\Pi$. In other words, if $s \equiv_{\Pi '} t$, then  $s \equiv_{\Pi} t$. We write $\Pi \leq \Pi '$. We also say that $\Pi$ is $\textit{coarser}$ than $\Pi '$, or that $\Pi '$ is $\textit{finer}$ than $\Pi$. 
    
    
    \item Prove that no matter what set $X$ we start with, (Partitions($X$), $\leq$) is always a partial order. Here, Partitions($X$) is the set of all partitions of $X$.
    
    
    
\end{enumerate}

\vspace{1.5in}



\section{Partition refinement on graphs}
Draw any graph on the set $\{1,2,3,4\}$. Take the five partitions which you found in problem 3, part 1, and call them $\Pi_1 , \dots , \Pi_5$. Then find refine($G$,$\Pi_1$), $\dots$, refine refine($G$,$\Pi_5$).

\vspace{1.5in}


\section{Partition refinement on graphs}
Let $G$ be a graph. Consider the function $$f : \text{partitions}(G) \rightarrow partitions(G)$$ given by $f(\Pi) = \text{refine}(G,\Pi)$.

\begin{enumerate}
    \item Show that $f$ is $\textit{monotone}$: if $\Pi \leq \Pi '$, then $f(\Pi) \leq f(\Pi ')$. You need to prove this for all graphs $G$ and all partitions of the set of nodes of $G$.
    
    \item Give an example of a graph $G$ and a partition $\Pi$ of its nodes such that $\Pi \nleqslant f(\Pi)$.
    
    \item Give an example of a graph $G$ and a partition $\Pi$ of its nodes such that $f(\Pi) \nleqslant \Pi$.
    
    \item Let $G$ be the graph shown in problem 4. Find a partition $\Pi$ such that $f(\Pi) = \Pi$.
    
    
\end{enumerate}
    
\end{document}
